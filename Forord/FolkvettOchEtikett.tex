\begin{intersong}
\flushleft\textbf{Lilla vett- och etikettskolan}

\flushleft\textbf{Klädkoder}\\
Vid många festtillfällen används en klädkod som dikterar vilken klädsel som ska användas. Klädkoden hjälper till att skapa en enhetlig och festlig stämning och att följa klädkoden visar att man respekterar festtillfället och festvärden. Att bryta mot klädkoden är, vid de flesta tillfällen, ändå ingen allvarlig förseelse.
\\
~\\
\flushleft\textbf{Högtidsdräkt}
Högtidsdräkt innebär frack och svarta byxor i samma tyg som fracken, som bärs upp med vit frackskjorta, vit frackfluga och vit frackväst tillsammans med svarta lackskor. För damerna gäller vid dessa tillfällen lång aftonklänning eller balklänning. Klänningen bör nå till golvet och vara av fint tyg, i festlig modell. Urringning är tillåten, ärmlösa modeller likaså. Bärs handskar ska de vara så långa så att de täcker armbågarna. Långa handskar behålls på till hälsning och dans, men tas av och läggs i knäet när du sitter till bords.
\\
~\\
Axlarna bör vara täckta vid högtidligare sammanhang. På årsfester är grundregeln att axlarna är täckta fram till konferensen.  En hembygdsdräkt eller folkdräkt eller plagg av religiös karaktär kan ersätta frack eller balklänning. Förtjänsttecken bärs med högtidsdräkt ifall det meddelas separat, och fästs på vänstra sidan, vid höjden av hjärtat.
\\
~\\
\flushleft\textbf{Mörk kostym}
Mörk kostym innebär enfärgad mörkgrå, mörkblå eller svart kostym med ljus kostymskjorta och slips samt svarta skor med långa, svarta strumpor. Diskreta ränder på kostymen är tillåtna. Damerna kan bära kjol eller klänning med längd under knät, festfina byxor är också okej. Mörk kostym kan ersätta högtidsdräkt för herrar. Hembygdsdräkt kan ersätta mörk kostym.
\\
~\\
\flushleft\textbf{Kavaj}
Kavaj är den minst strikta av vedertagna klädkoder. Männen kan ha mörk eller ljus kostym med färgad kostymskjorta utan slips. För kvinnor gäller det att klä upp sig för finare vardag, med knälång klänning eller kjol med jacka, eller fina byxor.
Med udda kavaj menas en klädsel där kavajen och byxan skiljer sig åt. För damerna innebär udda kavaj snygga byxor (inte jeans), knälång klänning eller kjol. Klädkoden ”snäppet snyggare” innebär oftast kavaj eller udda kavaj.
\\
~\\
\flushleft\textbf{Nationsband}
Inom nationerna, och också bland flera ämnesföreningar, används band för att visa sin tillhörighet inom studentvärlden. Vasa nations band har färgerna röd, gul och svart, och den svarta färgen ska ligga underst.
\\
~\\
Till nya medlemmar delas nationsband ut av inspektorn i samband med festligheterna på självständighetsdagen. Nationsband kan användas till städad klädsel. Nationsband ska inte vidröra bar hud. 
\\
~\\
Till kostym ska nationsbandet bäras som kortare version, över vänster kavajslag. Till frack fästs nationsbandet innanför fracken från höger till vänster sida. Knutet i rosett ska den svarta färgen vändas nedåt och till vänster. Rosetten fästs nära hjärtat. Till långklänning kan nationsbandet bäras som helband från vänster axel till höger sida.
\\
~\\
Nationsnålen med Vasa nations färger bärs som en symbol för nationstillhörighet. Nålen kan användas istället för nationsband till städad klädsel.

\flushleft\textbf{Förtjänsttecken}
Akademiska förtjänsttecken bärs till frack eller långklänning, i undantagsfall till mörk kostym. Akademiska förtjänsttecken fästs till vänster, nära hjärtat. Ett undantag är kamratskapstecknet som fästs mitt på nationsbandet. Förtjänsttecken bör inte vidröra bar hud. Nationens förtjänsttecken, aktivitetsmärken och kamratskapspris är akademiska förtjänsttecken och kan bäras vid akademiska fester och tillställningar då de hör till klädkoden.

\flushleft\textbf{Vid festen}
Bordsdamen sitter till höger om sin bordskavaljer. Vid bordet står man bakom sin stol tills värdinnan/värden är på plats och visar att man nu kan sätta sig.
\\
~\\
Efter att vi sjungit tillsammans skålar vi tillsammans. Bordsherren och bordsdamen skålar först med varandra, sen skålar man med personen som är på ens andra sida och sist skålar man med personen som sitter mitt emot. Vid finare fester är det brukligt med efterskål. Efterskål innebär att man, efter att man druckit ur glaset, igen lyfter glaset för skål men i omvänd ordning. Först mot personen mitt emot, sen till personen på ”sin andra sidan” och sen till bordsherren/bordsdamen.

\flushleft\textbf{Nationens värja}
Vasa nations värja, som används av sångledaren, fick nationen i gåva av vännationen Eesti Üli\~{o}pilaste Selts vid 100-års jubileet 26.2.2008. 
\\
~\\
Värjan, som är beprydd med nationens färger, används för att leda sången och för att kalla uppmärksamhet vid nationens fester och andra liknande tillställningar.
\\
~\\
Sångledaren eller vice sångledaren är ansvarig för värjan och dess användning under festen. Sångledaren, eller annan person som sångledaren gett möjlighet att leda sång, bör använda värjan med försiktighet och omsorg. Värjan är värdefull och ska användas respektfullt. 
\\
~\\
Sångledaren, eller den som av sångledaren anförtrotts värjan för att ta in en sång, ska leda sången ordentligt från början till slut. Festdeltagare bör respektera den som håller i värjan genom att lyssna och följa dennas anvisningar genast värjan höjts.

\flushleft\textbf{Överkurs}
Till värjan hör en del kommandon. Kommandona är latin eller förkortningar av latinska uttryck. Bli därför inte förskräckt om nån äldre (eller yngre) nationsmedlem börjar ropa underliga saker på nått underligt språk. 
\\
~\\
Verbum. Verbum betyder ord. Den som håller värjan meddelar genom att säga Verbum att hen ska prata och att sitzdeltagrana därför ska stilla sig, lyssna och visa respekt för den som talar.
\\
~\\
Verbum ex est. Den som håller värjan meddelar att hen talat klart.
\\
~\\
Cantus.  Cantus betyder att sjunga. Vi börjar sjunga.
\\
~\\
Cantus ex est. Sången är slut.
\\
~\\
Schmollis. Sis mihi mollis amicus. Ungefär ”var en nära vän till mig”. Är nationsmedlemmarnas svar på Cantus ex est. Schmollis sägs här istället för skål.  
\\
~\\
Fiducit. Ungefär ”det stämmer”. Gulisar ropar fiducit med gäll röst, medlemmar som studerat minst tretton terminer utan att ha blivit klara ropar fiducit med mörk stämma.

\end{intersong}