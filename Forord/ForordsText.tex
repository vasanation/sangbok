\begin{scriptsize}

\flushleft\textbf{Förord}\\
~\\
En akademisk sitz är ett påhitt som samlar gamla bekanta samtidigt som den ger möjlighet att stifta nya bekantskaper i sångens glada tecken. Sitzen är ett framgångsrikt koncept; i nästan alla akademiska föreningar förkommer sitzar av olika slag, även om detaljerna varierar mellan föreningar, universitet, städer och länder.
\\
~\\
Sången är ett centralt element under sitzen och därmed är även sångledarens roll viktig. Sångledarens uppgift är att läsa av festen, notera vilka sinnestämningar som råder och välja sångstil samt tempo utgående från detta. Sången ger törstiga festdeltagare möjligheten att fukta sina strupar och kan vara en räddare i nöden under eventuella stela diskussioner eller raggningsförsök.
\\
~\\
Sitzen och sången är inte enbart ett skäl till att dricka, men en god dryck blir om möjligt ännu godare efter en välkomponerad eller fyndig sång. Vasungavisor kan användas under nationsårets alla tillfällen. Sångboken ska fungera som en källa för inspiration till sång och glada stunder oberoende om du sitter i Vasahörnan efter en fest eller på en sommaräng uppe i Österbotten.
\\
~\\
Nationen är i ständig förändring och likaså sångboken. Den nya upplagan av Vasungavisor bär nationens sånghistoria med sig, samtidigt som nya sånger tagits in för att spegla nutidens sångkultur. Några sånger har även sponsorerats av personer som funnit sin favorit bland sångerna. I denna upplaga finns det även en liten vett- och etikettskola som kan läsas av både nya och erfarna festdeltagare. 
\\
~\\
Sångbokskommittén hoppas att du får många glädjerika stunder med Vasungavisor i din hand.
\\
~\\
Nu är det dags att vända blad, fatta glaset och sjunga av hjärtans lust!
\\
~\\
Ostrobotnia 18 november 2018
\\
~\\
Sångbokskommittén
\end{scriptsize}
