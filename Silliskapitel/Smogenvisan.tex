% Exempel på färdig-formaterad sång till VN:s
% sångbok 2018.

% Denna fil kan användas som sådan, bara verserna,
% namnen och annan rådata behöver bytas ur fälten.
% Tecknet "%" markerar en kommentar som helt och 
% hållet ignoreras av programmet som läser filen.

% Spara den färdiga filen som 
% 'SangnamnUtanMellanslagEllerSkander.tex'
% t.ex. blir "Vid En Källa" till 
% 'VidEnKalla.tex'
% Varje sång blir en egen fil.

\beginsong{Smögenvisan}[	% Författare
	sr={När jag var en ung Caballero}]	% Melodi
			% Alternativa
			% sångnamn
	
\beginverse*		% Börja vers
På västkusten hemma i Smögen 
där kallas jag allmänt för Gösta. 
Gösta för kärlek och solsken och sång 
för kärlek och solsken och sång, 
pling plong. 
\endverse			% Sluta vers

\beginverse*		% Börja vers
Jag åkte till storstaden Mora. 
Där träffa jag stans största kvinna. 
en kvinna för kärlek... 
\endverse			% Sluta vers

\beginverse*		% Börja vers
Hon sade att hon hette Ulla 
och att hon med mig ville prata. 
Prata om kärlek... 
\endverse			% Sluta vers

\beginverse*		% Börja vers
Hon fråga om jag ville titta 
på hennes rätt rymliga våning. 
En våning för kärlek... 
\endverse			% Sluta vers

\beginverse*		% Börja vers
Hon sa där var vackert om hösten 
och smekte de fylliga bolstren. 
Bolster för kärlek...
\endverse			% Sluta vers

\beginverse*		% Börja vers
Hon bjöd mig på kaffe och bulla.
Och efteråt börja vi sjunga.
Sjunga om kärlek… 
\endverse			% Sluta vers

\beginverse*		% Börja vers
När hemåt sen sakta jag lunka 
och stanna vid porten och rökte. 
Jag rökte för kärlek...
\endverse			% Sluta vers
\endsong			% Sluta sång
