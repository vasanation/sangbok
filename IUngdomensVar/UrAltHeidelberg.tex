\beginsong{Ur ``Alt Heidelberg''}[
	index={O, du studentens glada liv},
	index={O, jerum, jerum, jerum}]

\beginverse* 
O, du studentens glada liv,
varthän har du försvunnit
med dina muntra tidsfördriv,
din fröjd, som skyhögt brunnit?
Förgäves jag omkring mig ser,
jag finner dina spår ej mer.
\endverse
\beginchorus
O, jerum, jerum, jerum,
O, quae mutatio rerum.
\endchorus

\beginverse* 
{\setstretch{1.0}\textsubscript{\textit{\tiny{Teknologer}}}}
Var äro de som kunde allt,
blott ej sin ära svika,
{\setstretch{0.2}\textsubscript{\textit{\tiny{Ekonomer}}}}
som voro män av äkta halt,
och världens herrar lika?
De drogo bort från vin och sång
till vardagslivets tråk och tvång.
\endverse
\beginchorus
O, jerum...
\endchorus

\beginverse* 
\textsubscript{\textit{\tiny{Filosofer, samhällsvetare och andra vettiga}}}
Den ene vetenskap och vett
in i scholares mängder
\textsubscript{\textit{\tiny{Jurister}}}
den andre i sitt anlets svett
på paragrafer vränger
\textsubscript{\textit{\tiny{Teologer, psykologer}}}
En plåstrar själen som är skral
\textsubscript{\textit{\tiny{Medicinare, sjukskötare}}}
en lappar hop dess trasiga fodral
\endverse
\beginchorus
O, jerum...
\endchorus

\beginverse* 
Men hjärtat i studentens barm
kan ingen tid dock isa.
I allvar som i skämt, så varm
vill han sin ungdom prisa!
Det gamla skalet brustit har,
men kärnan finnes frisk dock kvar,
\endverse
\beginchorus
och vad vi än må mista
den skall dock aldrig brista!
\endchorus

\beginverse* 
Så sluten bröder fast vår krets
till glädjens värn och ära!
Trots allt vi tryggt och väl tillfreds
vår vänskap trohet svära.
Lyft bägarn högt och klinga vän!
De gamla gudar leva än 
\endverse
\beginchorus
bland skålar och pokaler,
bland skålar och pokaler!
\endchorus
\endsong


